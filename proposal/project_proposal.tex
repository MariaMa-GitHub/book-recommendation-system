\documentclass[fontsize=11pt]{article}
\usepackage{amsmath}
\usepackage[utf8]{inputenc}
\usepackage[margin=0.75in]{geometry}

\title{CSC111 Project Proposal: TODO FILL IN YOUR PROJECT TITLE HERE}
\author{Jayden Chiola-Nakai, Maria Ma, Kaiwen Zheng}
\date{Tuesday, March 16, 2021}

\begin{document}
\maketitle

\section*{Problem Description and Research Question}

\begin{enumerate}
    \item[1.]
    Give an overview of any background knowledge necessary for the reader to understand the problem you are studying.
    \item[2.]
    Provide context for the problem and motivate why you have chosen your project question/goal.
    \item[3.]
    Your project question/goal should be in bold; it should be fairly concise, but can be more than one sentence.
    \item[4.]
    200 – 400 words
\end{enumerate}

\section*{Computational Plan}

\begin{enumerate}
    \item[1.]
    Describe what kind of data your project will use to represent your chosen domain.
    
    Clearly explain how you will use trees and/or graphs to model a central part of the data. While your project can certainly use other data types (e.g., sets, lists), trees and/or graphs must play a prominent role in your program.
    
    If your project requires real-world data, provide a source for at least one relevant dataset you have found, and provide some sample data contained inside that dataset.

    \item[2.]
    Describe the kinds of computations you plan to perform, such as: building trees/graphs from a dataset or computation, data transformation/filtering/aggregation, computational models, and/or algorithms.
   
    Your proposed computations can be based on what you’ve done in lectures/tutorials/assignments, but should not simply replicate work you’ve already done.
    
    \item[3.]
    Explain how your program will display or report the results of your computations in a visual and/or interactive way. You don’t need to go into a lot of detail here, but it should be clear what you plan to do.
    
    \item[4.]
    In this part of your proposal, you should also describe one new library you intend to use, how you will use it, and why it is appropriate. Refer to specific functions, data types, and/or capabilities of the library that make it relevant for solving the problem you wish to solve. You don’t need to be an expert in this library, but should demonstrate that you’ve done more research into this library than just a quick Google search.

    \item[5.] 
    400 – 600 words
\end{enumerate}



\section*{References}

In this part of your proposal, you should also describe one new library you intend to use, how you will use it, and why it is appropriate. Refer to specific functions, data types, and/or capabilities of the library that make it relevant for solving the problem you wish to solve. You don’t need to be an expert in this library, but should demonstrate that you’ve done more research into this library than just a quick Google search.

% NOTE: LaTeX does have a built-in way of generating references automatically,
% but it's a bit tricky to use so we STRONGLY recommend writing your references
% manually, using a standard academic format like APA or MLA.
% (E.g., https://owl.purdue.edu/owl/research_and_citation/apa_style/apa_formatting_and_style_guide/general_format.html)

\end{document}
